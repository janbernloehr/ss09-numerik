\section{Numerische Integration}

\textit{Aufgabenstellung.}
Gegeben sind die Datensätze $(x_i,f(x_i))$, $i=0,1,\ldots,n$ von Werten einer
Funktion $f:[a,b]\to\R$.

Gesucht ist eine Approximation des Integrals,
\begin{align*}
\int\limits_a^b f(x)\dx.
\end{align*}
\textit{Anwendungen.}
\begin{itemize}
  \item Approximation von Integralen, die man nicht analytisch lösen kann.
\begin{bspn}
$\int\limits_0^1 e^{x^2}\dx$.\bsphere
\end{bspn}
  \item Integration von Funktionen, die nicht vollständig bekannt sind, sondern
  nur durch gemessene Daten oder eine im Computer implementierte Funktion.
\end{itemize}
Die allgemeine Sturktur einer numerischen Integration ist,
\begin{align*}
\int\limits_a^b f(x)\dx \approx \sum\limits_{i=0}^n \omega_i f(x_i),
\end{align*}
mit den \emph{Stützstellen} $x_i$ und \emph{(Integrations)-gewichten}
$\omega_i$. Eine solche Formel nennt man auch \emph{Quadraturformel}.

\subsection{Interpolatorische Quadraturformeln}

Naheliegend ist es, Interpolationstechniken zu nutzen, um die Quadraturformel
herzuleiten. Dazu integrieren wir das Interpolationspolynom zu den Daten
\begin{align*}
(x_i,f(x_i)),\qquad\text{für }i=0,\ldots,n.
\end{align*}

\subsubsection{Lagrange Darstellung}

Wir verwenden die Interpolationspolynome von Lagrange
\begin{align*}
p(x) = \sum\limits_{i=0}^n f(x_i)L_i(x),
\end{align*}
mit
\begin{align*}
L_i(x) = \prod\limits_{\atop{k=0}{k\neq i}}^n \frac{x-x_k}{x_i-x_k},\quad
L_i(x_k) = \delta_{ik}.
\end{align*}
Aufgrund der Linearität der Integration erhalten wir für das Integral,
\begin{align*}
\int\limits_a^b p(x)\dx = \sum\limits_{i=0}^n f(x_i)\underbrace{\int\limits_a^b
L_i(x)\dx}_{\omega_i}.
\end{align*}
\begin{bspn}
$[a,b]=[0,1]$, $n=2$ und $x_0=0$, $x_1=\frac{1}{2}$, $x_2=1$.
\begin{align*}
\omega_0 &= \int\limits_0^1 L_0(x)\dx
= \int\limits_0^1 \frac{(x-\frac{1}{2})(x-1)}{\frac{1}{2}\cdot 1}\dx\\
&= \int\limits_0^1 2\left( x^2 -x-\frac{1}{2}x+\frac{1}{2}\right)\dx
= 2\left(\frac{1}{3} - \frac{3}{4} + \frac{1}{2}\right)
= \frac{1}{6},\\
\omega_1 &= \int\limits_0^1 L_0(x)\dx
= \int\limits_0^1 \frac{x(x-1)}{\frac{1}{2}\cdot \frac{-1}{2}}\dx\\
&= \int\limits_0^1 -4\left( x^2-x\right)\dx
= -\frac{4}{3} + 2 = \frac{2}{3},\\
\omega_2 &= \int\limits_0^1 L_0(x)\dx
= \int\limits_0^1 \frac{x(x-\frac{1}{2})}{1\cdot \frac{1}{2}}\dx\\
&= \int\limits_0^1 2\left( x^2-\frac{1}{2}x\right)\dx
= \frac{2}{3} - \frac{1}{2} = \frac{1}{6}.
\end{align*}
Für die Quadraturformel erhalten wir
\begin{align*}
Q(f) = \frac{1}{6}\left(f(0)+4f\left(\frac{1}{2}\right) + f(1) \right).\bsphere
\end{align*}
\end{bspn}

\begin{bemn}[Ergebnis.]
Die interpolatorische Quadraturformel zu dem Integrationsintervall $[a,b]$ und 
den Stützstellen $x_0,\ldots,x_n$ ist,
\begin{align*}
Q(f) = \sum\limits_{i=0}^n \omega_i f(x_i),\tag{1}
\end{align*}
mit
\begin{align*}
\omega_i = \int\limits_a^b \prod\limits_{\atop{j=0}{j\neq i}}^n
\frac{x-x_j}{x_i-x_j}\dx.\tag{2}
\end{align*}
\end{bemn}

\subsubsection{Newton-Cotes-Formeln}
Newton-Cotes-Formeln sind interpolatorische Quadraturformeln mit
\textit{äquidistanten Stützstellen}, d.h. $x_i-x_{i-1} = h$.

\begin{defnn}
Man unterscheidet zwischen \emph{abgeschlossenen} Newton-Cotes-Formeln
\begin{align*}
x_0 = a,\; x_n = b,\; x_i = a+ i\cdot h,\; h= \frac{b-a}{n},
\end{align*}
und \emph{offenen} Newton-Cotes-Formeln
\begin{align*}
x_1 = a+\frac{h}{2},\; x_n = b-\frac{h}{2},\; x_i = a+
\frac{2i-1}{2}h,\;h=\frac{b-a}{n}.
\end{align*}
Die Quadraturformel hat die Form von (1),
\begin{align*}
Q(f) = \sum\limits_{i=1}^n \omega_i f(x_i),
\end{align*}
die $\omega_i$ sind durch (2) gegeben.\fishhere
\end{defnn}


\begin{figure}
% Für $[a,b]=[0,1]$ sind die Integrationsgewichte der abgeschlossenen Formeln

\begin{tabular}[h]{c|cl|cl}
$n$ & $\omega_i$ abgeschlossen & & $\omega_i$ offen \\\hline
$1$ & $\frac{1}{2},\;\frac{1}{2}$ & Trapezregel & $1$ & Mittelpunktsregel\\
$2$ & $\frac{1}{6},\;\frac{2}{3},\;\frac{1}{6}$ & Simpson-Regel
& $\frac{1}{2},\;\frac{1}{2}$\\
$3$ & $\frac{1}{8},\;\frac{3}{8},\;\frac{3}{8},\;\frac{1}{8}$ &
$\frac{3}{8}$-Regel & $\frac{3}{8},\;\frac{1}{4},\;\frac{3}{8}$
\end{tabular}
\caption{Integrationsgewichte der abgeschlossenen und offenen Formeln für
$[a,b]=[0,1]$}
% und der offenen Formeln
% 
% \begin{tabular}[h]{c|c|l}
% $n$ & $\omega_i$ \\\hline
% $1$ & $1$ & Mittelpunktsregel \\
% $2$ & $\frac{1}{2},\;\frac{1}{2}$ & \\
% $3$ & $\frac{3}{8},\;\frac{1}{4},\;\frac{3}{8}$ &
% \end{tabular}
\end{figure}


\begin{figure}[h!tbp]
\centering
\begin{pspicture}(-0.2,-1.2221875)(3.141875,1.2021875)
\psline{->}(0.62625,-0.9578125)(0.62625,0.8021875)
\psline{->}(0.50625,-0.8378125)(2.98625,-0.8378125)
\pspolygon[fillstyle=solid,fillcolor=glightgray](0.62625,-0.1778125)(2.72625,0.7021875)(2.70625,-0.8378125)(0.62625,-0.8378125)
%\psline[fillstyle=solid,fillcolor=glightgray](0.62625,0.1821875)(2.70625,0.1821875)
%TODO: glightblue
\psbezier[linecolor=darkblue](0.58625,-0.2978125)(0.92625,1.1821876)(2.22625,-0.9578125)(2.76625,0.8421875)

\rput(3.0117188,-1.0078125){\color{gdarkgray}$x$}

\rput(0.231875,0.8121875){\color{gdarkgray}$f(x)$}

\rput(2.673125,-1.0678124){\color{gdarkgray}$1$}

\rput(0.50328124,-1.0678124){\color{gdarkgray}$0$}
\end{pspicture} 
\begin{pspicture}(-0.2,-1.2221875)(3.141875,1.2021875)
\psline{->}(0.62625,-0.9578125)(0.62625,0.8021875)
\psline{->}(0.50625,-0.8378125)(2.98625,-0.8378125)
\pspolygon[fillstyle=solid,fillcolor=glightgray](0.62625,0.1821875)(2.72625,0.1821875)(2.70625,-0.8378125)(0.62625,-0.8378125)
%\psline[fillstyle=solid,fillcolor=glightgray](0.62625,0.1821875)(2.70625,0.1821875)
%TODO: glightblue
\psbezier[linecolor=darkblue](0.58625,-0.2978125)(0.92625,1.1821876)(2.22625,-0.9578125)(2.76625,0.8421875)

\rput(3.0117188,-1.0078125){\color{gdarkgray}$x$}

\rput(0.231875,0.8121875){\color{gdarkgray}$f(x)$}

\rput(2.673125,-1.0678124){\color{gdarkgray}$1$}

\rput(0.50328124,-1.0678124){\color{gdarkgray}$0$}
\end{pspicture}
\caption{Trapetz- und Mittelpunktsregel.}
\end{figure}

\begin{bemn}
Es genügt die Quadraturformeln des Integrationsgebiets $[0,1]$ zu kennen. Die
Formeln für alle weiteren Gebiete folgen aus der Transformationsformel für Integrale,
\begin{align*}
\int\limits_a^b f(x)\dx &\overset{x=a+(b-a)y}{=} \int\limits_0^1 f\left(a+(b-a)y
\right)(b-a)\dy\\
&\approx \sum\limits_{i=0}^n \underbrace{(b-a)\omega_i}_{\tilde{\omega}_i}
f\left(\underbrace{a+(b-a)y}_{\tilde{x}_i} \right).\maphere
\end{align*}
\end{bemn}

\begin{bemn}[Probleme.]
Bei der Verwendung von Newton-Cotes-Formeln können für großes $n$ ($n\ge 8$)
\textit{negative} Integrationsgewichte auftreten. Dies kann zu Auslöschung
bzw. zu numerischer Instabilität führen. 

Außerdem konvergiert das Interpolationspolynom nicht notwendigerweise gegen die
approximierte Funktion.
\end{bemn}

\subsubsection{Zusammengesetzte Newton-Cotes-Formeln}

Die Idee ist, das Integrationsgebiet $[a,b]$ in Teilintervalle $[y_{i-1},y_i]$,
$i=1,\ldots,m$ mit $a=y_0<y_1<\ldots<y_n =b$ zu unterteilen, auf denen das
jeweilige Interpolationspolynom die Funktion gut approximiert.
Verwende nun für jedes Teilintervall eine Newton-Cotes-Formel.
\begin{bspn}
\textit{Zusammengesetzte Trapetzregel}. Setze
\begin{align*}
y_i = a+ih,\quad i=0,\ldots,m,\quad h=\frac{b-a}{m}.
\end{align*}
Für das Integral folgt,
\begin{align*}
\int\limits_{a}^b f(x)\dx &= \sum\limits_{i=1}^m
\int\limits_{y_{i-1}}^{y_i} f(x)\dx
\approx \sum\limits_{i=1}^m \frac{y_i-y_{i-1}}{2}\left(f(y_{i-1}) + f(y_i)
\right)\\
&= \frac{h}{2}f(a) + h\sum\limits_{i=1}^{m-1} f(a+ih) + \frac{h}{2}f(b).
\end{align*}
Die zusammengesetzte Integrationsformel hat also die Form,
\begin{align*}
T_m(f) = h\left[\frac{1}{2}f(a) + \sum\limits_{i=1}^{m-1} f(a+ih) +
\frac{1}{2}f(b) \right],\qquad h=\frac{b-a}{m}.
\end{align*}
\begin{figure}[h!tbp]
\centering
\begin{pspicture}(-0.2,-1.1121875)(5.041875,1.1021875)
\psline{->}(0.62625,-0.8478125)(0.62625,0.9121875)
\psline{->}(0.50625,-0.7278125)(4.98625,-0.7278125)

\pspolygon(0.62625,0.1121875)(1.08625,-0.1478125)(1.92625,-0.0278125)(2.78625,0.4721875)(3.74625,0.7521875)(4.68625,0.3121875)(4.68625,-0.7278125)(0.62625,-0.7278125)
\psbezier[linecolor=darkblue](0.62625,0.1121875)(1.04625,-0.4078125)(1.98625,-0.1278125)(2.56625,0.3121875)(3.14625,0.7521875)(4.16625,1.0721875)(4.74625,0.2321875)

\rput(0.231875,0.9221875){\color{gdarkgray}$f(x)$}
%\rput(0.50328124,-0.9578125){\color{gdarkgray}$0$}
\rput(4.911719,-0.9578125){\color{gdarkgray}$x$}

\psline(1.08625,-0.6478125)(1.08625,-0.8878125)
\psline(1.92625,-0.6478125)(1.92625,-0.8878125)
\psline(2.78625,-0.6478125)(2.78625,-0.8878125)
\psline(3.74625,-0.6478125)(3.74625,-0.8878125)
\psline(4.68625,-0.6478125)(4.68625,-0.8878125)
\end{pspicture} 
\caption{Zusammengesetzte Trapetzregel.}
\end{figure}
Wir sehen sowohl optisch als auch in den Fehlern ein deutlich besseres
Konvergenzverhalten.\bsphere
\end{bspn}

\begin{bspn}
\textit{Zusammengesetzte Simpsonregel}. Setze
\begin{align*}
&y_i = a+i2h,\quad i=0,\ldots,\frac{m}{2},\quad \frac{m}{2}\in\N,\quad
h=\frac{b-a}{m},\\
&x_i = a+ih,\quad i=0,\ldots,m.
\end{align*}
\begin{align*}
s_m(f) &= h\bigg[\frac{1}{6}f(x_0) + \frac{2}{3}f(x_1) + \frac{1}{6}f(x_2)
+
 \frac{1}{6}f(x_2) + \frac{2}{3}f(x_3) + \frac{1}{6}f(x_4) \\ &\qquad+ \ldots
 + \frac{1}{6}f(x_{m-2}) + \frac{2}{3}f(x_{m-1})+\frac{1}{6}f(x_n)\bigg]\\
&= \frac{h}{6}\left(f(a)+4\sum\limits_{i=1}^{m/2} f(x_{2i}-1) +
2\sum\limits_{i=1}^{m/2-1} f(x_{2i}) + f(b)\right).
\end{align*}
Man kann von der zusammengesetzten Simpsonsformel ein deutlich schnelleres
Konvergenzverhalten erwarten als von der zusammengesetzten Trapetzregel. Wir
werden dies später bei der Fehleranalyse sehen.\bsphere
\end{bspn}

\subsubsection{Approximationseigenschaften}

Durch Anwendung von Satz \ref{prop:3.4}, der Fehlerabschätzung für
Interpolationspolynome, erhalten wir auch eine Fehlerabschätzung für die
Quadraturformel:
\begin{prop}
\label{prop:4.1}
Sei $Q(f) = \sum\limits_{i=0}^m \omega_i f(x_i)$ eine Quadraturformel mit
\begin{align*}
Q(p) = \int\limits_a^b p(x)\dx,\qquad\text{ für alle }p\in\PP_m. 
\end{align*}
Dann gilt für $f\in C^{m+1}([a,b])$
\begin{align*}
\abs{\int\limits_a^b f(x)\dx-Q(f) } \le \frac{1}{(m+1)!} \norm{f^{(m+1)}}_\infty
\int\limits_a^b \prod\limits_{j=0}^m (x-x_j)\dx.\fishhere
\end{align*}
\end{prop} 
\begin{proof}
Sei $p$ das Interpolationspolynom zu
\begin{align*}
(x_i,f(x_i)),\qquad i = 0,\ldots,m,
\end{align*}
d.h. $p\in\PP_m$. Für die Quadraturformel gilt nach Vorassetzung,
\begin{align*}
Q(f) = Q(p) = \int\limits_a^b p(x)\dx.
\end{align*}
Mit Satz \ref{prop:3.4} folgt somit,
\begin{align*}
&\abs{\int\limits_a^b f(x)\dx - Q(f)} = \abs{\int\limits_a^b f(x) - p(x)\dx}\\
&\le \int\limits_a^b \frac{1}{(m+1)!}\abs{f^{(m+1)}(\xi(x))}\abs{\prod
\limits_{j=0}^m (x-x_j)}\dx\\
&\le \frac{1}{(m+1)!} \norm{f^{(m+1)}}_\infty \int\limits_a^b
\prod\limits_{j=1}^m \abs{x-x_j}\dx.\qedhere
\end{align*}
\end{proof}

\begin{bspn}
\textit{Trapezregel}
\begin{align*}
Q(f) = \frac{b-a}{2}\left(f(a) + f(b) \right).
\end{align*}
Hier ist $m=1$, $x_0=a$, $x_1=b$,
\begin{align*}
&\int\limits_a^b \abs{(x-x_0)(x-x_1)}\dx = \int\limits_a^b \abs{(x-a)(x-b)}\dx
= \int\limits_a^b (x-a)(b-x)\dx \\ 
&= \int\limits_0^{b-a} (b-a-x)\dx = 
\left[-\frac{1}{3}x^3 + \frac{b-a}{2}x^2 \right]_0^{b-a} = \frac{1}{6}(b-a)^3.\\
\Rightarrow & \abs{\int\limits_a^b f(x)\dx - Q(f)} \le
\frac{(b-a)^3}{12}\norm{f''}_\infty.
\end{align*}
\textit{Zusammengesetzte Trapezregel} mit Schrittweite $h=\frac{b-a}{n}$,
$x_j=a+j h$.
\begin{align*}
\abs{\int\limits_a^b f(x)\dx-\Gamma_n(f)} = \abs{\sum\limits_{i=1}^n
\int\limits_{x_{i-1}}^{x_i}f(x)\dx - Q_i(f)},
\end{align*}
wobei $Q_i$ die Trapezregel für das entsprechende Intervall $[x_{i-1},x_i]$
darstellt.
%TODO: Jeanine fragen, \le (x-x_i)^3
\begin{align*}
&\le \sum\limits_{i=1}^n \frac{(x_i-x_{i-1})^3}{2}\norm{f''}_\infty \le
n\frac{h^3}{12}\norm{f''}_\infty = \frac{b-a}{h}\frac{h^3}{12}\norm{f''}_\infty
\\ &= \frac{b-a}{12}h^2\norm{f''}_\infty.\bsphere
\end{align*}
\end{bspn}
\begin{bspn}
\textit{Simpsonregel}
\begin{align*}
Q(f) = \frac{b-a}{2}\left(f(a) + 4f\left(\frac{a+b}{2}\right) + f(b) \right).
\end{align*}
Hier ist $m=2$, $x_0=a$, $x_1 = \frac{a+b}{2}$, $x_2 =b$,
\begin{align*}
\int\limits_a^b \abs{\prod\limits_{j=0}^2 (x-x_j)}\dx 
&= \int\limits_a^b \abs{(x-a)\left(x-\frac{a+b}{2}\right)(x-b)}\dx\\
&= 2\int\limits_a^{\frac{a+}{2}}
\abs{(x-a)\left(x-\frac{a+b}{2}\right)(x-b)}\dx\\
&= 2\int\limits_0^{\frac{b-a}{2}}
\abs{x\left(x-\frac{b-a}{2}\right)\left(x-b-a\right)}\dx \\
&=2\left[\frac{x^4}{2}-\frac{3}{2}(b-a)\frac{x^3}{3} +
\frac{(b-a)^2}{2}\frac{x^2}{2} \right]_{0}^{\frac{b-a}{2}}\\
&= 2\frac{1}{2^4}\left[\left(\frac{1}{4} - 1 + 1\right)(b+a)^4 \right] =
\frac{1}{32}(b-a)^4\\
\Rightarrow & \abs{\int\limits_a^b f(x)\dx -Q(f)} \le
\frac{(b-a)^4}{192}\norm{f'''}_\infty.
\end{align*}
Zusammengesetze Simpsonregel mit Schrittweite $h=\frac{b-a}{n}$.
\begin{align*}
\abs{\int\limits_a^b f(x)\dx - S_n(f)} \le n \frac{h^4}{192}\norm{f'''}_\infty
= \frac{b-a}{192}h^3\norm{f'''}_\infty.
\end{align*}
Die Abschätzung ist \textit{nicht optimal}. Numerische Ergebnisse lassen
einen Faktor $\sim h^4$ erwarten. Wir können unsere Fehlerabschätzung
verbessern, wenn wir berücksichtigen, dass die Simpsonregel auch Polynome vom
Grad 3 exakt integriert.\bsphere
\end{bspn}

\begin{defn}
\label{defn:4.2}
Eine Quadraturformel der Form
\begin{align*}
Q(f) = \sum\limits_{i=0}^n \omega_i f(x_i)
\end{align*}
heißt \emph{symmetrisch} genau dann, wenn
\begin{align*}
x_{m-j} - x_{j} = a+b,\quad\text{und }\omega_{m-j} = \omega_j,\quad
j=0,\ldots,m.
\end{align*}
\textit{Spezialfall}. $[a,b]=[-1,1]$. Dann heißt
\begin{align*}
Q(f) = \sum\limits_{i=0}^m \omega_i f(x_i)
\end{align*}
symmetrisch genau dann, wenn
\begin{align*}
x_{m-j} = -x_j,\quad \omega_{m-j} = \omega_j.\fishhere
\end{align*}
%TODO: Skizze.
\end{defn}
Es gilt dann für $k$ ungerade, dass
\begin{align*}
\int\limits_{-1}^1 x^k\dx = 0.
\end{align*}
\begin{align*}
Q(x\mapsto x^k) = \sum\limits_{i=0}^m \omega_i x_i^k =
\sum\limits_{i=0}^{\nrm{\frac{m-1}{2}}} \omega_i \underbrace{\left(x_i^k +
(-x_i)^k \right)}_{=0}
+
\begin{cases}
0, & m\text{ ungerade},\\
\omega_{\frac{m}{2}}\cdot \underbrace{x_{\frac{m}{2}}^k}_{0^k} , & m\text{
gerade}.
\end{cases}
\end{align*}
\begin{corn}
$Q$ integriert Polynome der Art $p(x)=x^k$, $k$ ungerade, exakt (unabhängig vom
Grad).\fishhere
\end{corn}

\begin{defn}
\label{defn:4.3}
Eine Quadraturformel $Q$ für das Intervall $[a,b]$ heißt \emph{exakt} auf
$\PP_k$ genau dann, wenn
\begin{align*}
Q(p) = \int\limits_a^b p(x)\dx,\qquad \forall p\in\PP_k.\fishhere
\end{align*}
\end{defn}

\begin{prop}
\label{prop:4.4}
Eine symmetrische Quadraturformel, die auf $\PP_{2l}$, $l\in \N_0:=\N\cup\{0\}$
exakt ist, ist auch auf $\PP_{2l+1}$ exakt.\fishhere
\end{prop}
\begin{proof}
Sei $Q(f) = \sum\limits_{i=0}^m \omega_i f(x_i)$ eine symmetrische
Quadraturfromel auf $[a,b]$. Wir betrachten zunächst den Spezialfall
\begin{align*}
&q(x) = \left(x - \frac{a+b}{2}\right)^{2l+1},\\
&\int\limits_{a}^b q(x)\dx = \int\limits_a^b \left(x -
\frac{a+b}{2}\right)^{2l+1} \dx = \int\limits_{-\frac{b-a}{2}}^{\frac{b-a}{2}}
y^{2l+1}\dy = 0.
\end{align*}
Daher ist
\begin{align*}
Q(q) = \sum\limits_{i=0}^m\underbrace{\omega_i}_{\omega_{m-i}}
\underbrace{\left( x_i -
\frac{a+b}{2} \right)^{2l+1}}_{-\left(x_{m-i}-\frac{a+b}{2}\right)} = 0,
\end{align*}
aufgrund der Symmetrie.

Für $p\in\PP_{2l+1}$ existiert ein $\alpha\in\R$ und $r\in\PP_{2l}$ mit $p =
\alpha q + r$.
\begin{align*}
\Rightarrow  Q(p) &= \sum\limits_{i=0}^m \omega_i\left(\alpha q +r \right)(x_i)
= \alpha\sum\limits_{i=0}^m \omega_i q(x_i) + \sum\limits_{i=0}^m \omega_i
r(x_i)\\
&= \alpha \int\limits_a^b q(x)\dx + \int\limits_a^b r(x)\dx = 
\int\limits_a^b p(x)\dx.\qedhere
\end{align*}
\end{proof}
\begin{corn}
Eine Newton-Cotes-Formel
\begin{align*}
Q(f) = \sum\limits_{i=0}^m \omega_i f(x_i),
\end{align*}
der Stufe $m$ ist
\begin{enumerate}[label=(\roman{*})]
  \item symmetrisch,
  \item  exakt auf $
\begin{cases}
\PP_m, & \text{für }m\text{ ungerade},\\
\PP_{m+1}, & \text{für }m\text{ gerade}.\fishhere
\end{cases}
$
\end{enumerate}
\end{corn}
\begin{proof}
Der Beweis der Symmetrie sei als Übung überlassen.\qedhere
\end{proof}
Insbesondere ist die Simpsonformel exakt auf $\PP_3$.

\begin{prop}[Fehlerdarstellung von Peano]
\label{prop:4.5}
Sei $Q(f)=\sum\limits_{i=0}^m \omega_if(x_i)$ eine auf $\PP_k$ exakte
Integrationsformel für $[a,b]$. Dann gilt für $f\in C^{k+1}([a,b])$
\begin{align*}
&Q(f) - \int\limits_a^b f(x)\dx = \int_a^b K(t)f^{(k+1)}(t)\dt,\\
&K(t) = \frac{1}{k!}\left[\sum\limits_{i=0}^m \omega_i (x_i-t)_+^k -
\int\limits_a^b (x-t)_+^k \dx \right]\\
&\qquad= \frac{1}{k!}\left[\sum\limits_{i=0}^m
Q\left(x\mapsto (x-t)_+^k\right) - \int\limits_a^b (x-t)_+^k \dx
\right]\fishhere
\end{align*}
\end{prop}
\begin{proof}
Taylorentwicklung mit Integralrestglied ergibt,
\begin{align*}
f(x) &= \sum\limits_{l=0}^k \frac{f^{(l)}(a)}{l!}(x-a)^l + \int\limits_a^x
\frac{(x-t)^k}{k!}f^{(k+1)}(t)\dt \\&=
\sum\limits_{l=0}^k \frac{f^{(l)}(a)}{l!}(x-a)^l + \int\limits_a^b
\frac{(x-t)_+^k}{k!}f^{(k+1)}(t)\dt.
\end{align*}
Wenden wir darauf die Integrationsformel an,
\begin{align*}
Q(f) &= \sum\limits_{l=0}^k \frac{f^{(l)}(a)}{l!}\underbrace{\sum\limits_{i=0}^m
\omega_i (x_i-a)^l}_{Q((x-a)^l)} + \int\limits_a^b \sum\limits_{i=0}^m \omega_i
(x_i-t)_+^k \frac{f^{(k+1)}(t)}{k!}\dt\\
&= \sum\limits_{l=0}^k \frac{f^{(l)}(a)}{l!}\int\limits_a^b (x-a)^l\dx +
\int\limits_a^b \sum\limits_{i=0}^m \omega_i (x_i-t)_+^k
\frac{f^{(k+1)}(t)}{k!}\dt.
\end{align*}
Das Integral über $f$ können wir auch darstellen als,
\begin{align*}
\int\limits_a^b f(x)\dx = \sum\limits_{l=0}^k
\frac{f^{(l)}(a)}{l!}\int\limits_a^b (x-a)^l\dx +
\int\limits_a^b\int\limits_a^b \frac{(x-t)_+^k}{k!}f^{(k+1)}(t)\dx\dt.
\end{align*}
Somit ergibt sich die Behauptung,
\begin{align*}
Q(f) - \int\limits_a^b f(x)\dx = 
\int\limits_a^b\frac{1}{k!}\left[ \sum\limits_{i=0}^m \omega_i (x_i-t)_+^k
 - \int\limits_a^b
(x-t)_+^k\dx\right]f^{(k+1)}(t)\dt.\qedhere
\end{align*}
\end{proof}

\begin{bspn}
\textit{Trapezregel.}
\begin{align*}
Q(f) &= \frac{b-a}{2}\left(f(a) + f(b) \right),\quad k=1\\
\Rightarrow K(t) &= \frac{b-a}{2}\left[(a-t)_+ + (b-t)_+\right] -
\int\limits_a^b (x-t)_+\dx\\
&\overset{\text{für }t\in(a,b)}{=}
\frac{b-a}{2}(b-t)- \int\limits_t^b (x-t)\dx
= \frac{b-t}{2}\left[b-a-(b-t)\right] \\& = \frac{1}{2}(b-t)(t-a).
\end{align*}
Wegen $K(t)\ge0$ für $t\in[a,b]$ folgt,
\begin{align*}
\int\limits_a^b K(t)f''(t)\dt &= f''(\xi)\int\limits_a^b K(t)\dt
= f''(\xi)\int\limits_0^{b-a} \frac{y(b-a-y)}{2} \dy\\
&= f''(\xi) \frac{\frac{1}{2}by^2 - \frac{1}{2}ay^2 -
\frac{1}{3}y^3}{2}\bigg|_0^{b-a}\\
&= f''(\xi) \frac{3b(b-a)^2 - 3a(b-a)^2 -
2(b-a)^3}{12}\\
&=\frac{f''(\xi)}{12}(b-a)^3
\end{align*}
\textit{Simpsonregel.}
\begin{align*}
&Q(f) = \frac{b-a}{6}\left(f(a) + 4\frac{f(a+b)}{2} + f(b) \right)\\ 
\Rightarrow &K(t) = 
\begin{cases}
\frac{(t-a)^3}{72}(a+2b-3t), & a\le t\le \frac{a+b}{2},\\
\frac{(b-t)^3}{72}(3t-2a-b), & \frac{a+b}{2}\le t\le b,
\end{cases}\\
&Q(f) -\int\limits_a^b f(x)\dx = \frac{(b-a)^5}{2880} f^{(4)}(\xi)\bsphere
\end{align*}
\end{bspn}

\begin{bemn}[Bemerkungen.]
\begin{enumerate}[label=\arabic{*}.)]
\item Falls $Q$ exakt auf $\PP_0$, also
\begin{align*}
\sum\limits_{i=0}^m \omega_i = b-a = \int_a^b 1\dx,
\end{align*}
und $\omega_i \ge 0$ für $i=0,\ldots,m$, dann
\begin{align*}
\abs{K(t)} &\le \frac{1}{k!}\left[ \sum\limits_{i=0}^m \omega_i (x_i-t)_+^k +
\int\limits_a^b (x-t)_+^k \dx \right]\\
&\le  \frac{1}{k!}\left[ \sum\limits_{i=0}^m \omega_i (b-a)^k +
\int\limits_a^b (b-a)^k \dx \right] \\ &\le
\frac{2}{k!}(b-a)^{k+1}
\end{align*}
Einsetzen in die Fehlerabschätzung ergibt,
\begin{align*}
\abs{Q(f)-\int\limits_a^b f(x)\dx} \le
\int\limits_a^b \abs{K(t)}\abs{f^{(k+1)}(t)}\dt
\le \norm{f^{(k+1)}}_\infty \frac{2}{k!}(b-a)^{k+2}.
\end{align*}
Die Abschätzung ist jedoch nicht so genau, hier erhalten wir für die Konstante
$\frac{2}{3}$ und dies ist weit weg von $\frac{1}{2280}$
\item
Falls $K(t)$ \textit{keinen} Vorzeichenwechsel auf $[a,b]$ hat, dann gilt
\begin{align*}
\int\limits_a^b K(t)^{(k+1)}(t)\dt = f^{(k+1)}(\xi)\int\limits_a^b K(t)\dt
\end{align*}
mit $\xi\in[a,b]$.\maphere
\end{enumerate}
\end{bemn}

\subsubsection{Anwendung auf zusammengesetzte Quadraturformeln}

Betrachte das Integrationsintervall $[a,b]$ mit der Unterteilung
\begin{align*}
[y_{j-1},y_j],\quad j=1,\ldots,n,
\end{align*}
mit $y_j = a+jh$ und Schrittweite $h=\frac{b-a}{h}$.

\begin{defnn}
Eine \emph{zusammengesetzte Quadraturformel} ist gegeben durch,
\begin{align*}
Q(f) = \sum\limits_i^n Q_i(f),
\end{align*}
wobei die $Q_i$ Quadraturformeln für $[y_{i-1},y_i]$ sind.\fishhere
\end{defnn}

\begin{bemn}[Voraussetzung.]
$Q_j$ ist exakt auf $\PP_k$.
\begin{align*}
\Rightarrow \abs{Q(f)-\int\limits_a^b f(x)\dx} &\le \sum\limits_{i=1}^n
\abs{Q_i(f)-\int\limits_{y_{i-1}}^{y_i} f(x)\dx}
\le nCh^{k+2}\norm{f^{(k+1)}}_\infty \\ &= (b-a)\norm{f^{(k+1)}}_\infty
h^{k+1}
\end{align*}
Dabei hängt $C$ von $Q$ ab, es gilt
\begin{align*}
C \le \frac{2}{k!}.
\end{align*}
Die obere Schranke muss nicht gut sein - wie wir bei der Simpsonregel gesehen
haben - sie funktioniert dafür aber allgemein.
\end{bemn} 

\subsection{Gauß-Quadratur}

Bisher waren die Stützstellen $(x_i)$ einer Quadraturformel
\begin{align*}
Q(f) = \sum\limits_{i=0}^m \omega_i f(x_i)
\end{align*}
fest vorgegeben.

\begin{bemn}[Frage:]
Kann man die Genauigkeit einer Quadraturformel verbessern, indem man die
Stützstellen $x_i$ geschickt wählt?
\end{bemn}

\begin{defnn}[Hilfsmittel]
Sei $[a,b]$, $a<b$ ein Intervall. Dann wird auf $\PP_m$ durch
\begin{align*}
\lin{p,q} = \int\limits_a^b p(x)q(x)\dx.
\end{align*}
ein \emph{Skalarprodukt} definiert, d.h.
\begin{enumerate}[label=(\roman{*})]
\item $\lin{p+q,r} = \lin{p,q}+\lin{q,r}$,\quad $\forall p,q,r\in\PP_m$\\
$\lin{\alpha p,r} = \alpha\lin{p,r}$,\quad $\forall p,r\in\PP_m$, $\alpha\in\R$,
\item $\lin{p,q}=\lin{q,p}$,\quad $\forall p,q\in\PP_m$,
\item $\lin{p,p} \ge 0$,\quad $\forall p\in\PP_m$.\fishhere
\end{enumerate}
\end{defnn}
Das Skalarprodukt und die Orthogonalitätsbeziehung hängen vom Intervall $[a,b]$
ab, d.h. $p\bot q$ auf $[a,b]$ erzwingt nicht $p\bot q$ auf $[c,d]$.